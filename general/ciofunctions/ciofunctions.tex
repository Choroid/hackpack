\section{C IO Functions}
Occasionally it is far easier to use the C IO functions to meet an output spec. It is also possible to set the precision and width options via numbers after the present, but before the specifier.  The general form of a specifier is:

\begin{lstlisting}[label=format code format,caption=Format Codes for printf()]
%[flags][width][.precision][length]specifier
\end{lstlisting}

\begin{table}[h]
	\caption{Format Specifier Codes\cite{cplusplus}}
	\begin{tabularx}{\textwidth}{|l|X|l|} \hline
		Format Code &   Output              &   Example     \\ \hline
		d           &   Signed Int          &   314         \\
		u           &   Unsigned Int        &   314         \\
		o           &   Unsigned Octal      &   472         \\
		x           &   Unsigned hex        &   13a         \\
		X           &   UNSIGNED HEX        &   13A         \\
		f           &   floating point      &   3.140000    \\
		e           &   Scientific notation &   3.140000e+00\\
		c           &   character           &   A           \\
		s           &   string              &   ACM         \\
		p           &   pointer address     &   0x40060c    \\
		l           &   Used with other specifiers to indicate a long & 314 \\
		\%\%        &   Prints a literal \% &   \%          \\
		\hline
	\end{tabularx}
\end{table}

\begin{table}[h]
	\caption{Modifier Flags \cite{cplusplus}}
	\begin{tabularx}{\textwidth}{|l|X|l|} \hline
		Format Code &   Output                  &   Example    \\ \hline
		-           &   Left-justify            &   314        \\
		+           &   Force-sign character    &   +314       \\
		\#          &   Show prefix             &   0x13a      \\
			  &   Show decimal point      &   314.       \\
		0           &   Left pad field with 0   &   0314       \\
		\hline
	\end{tabularx}
\end{table}

\subsection{Examples}
\acmlisting[language=c++]{./general/ciofunctions/ciofunctions.cpp}

\section{Sieve of Eratosthenes}\index{sieve!Eratosthenes}\index{primes}
The sieve of Eratosthenes is a simple, yet effective algorithm for generating primes less than around 10 million.
It works by iteratively eliminating (''sifting'') multiples of primes.
Numbers that are left are prime.
The algorithm runs in $O(n\log\log n)$ time.

\subsection{Applications}
\begin{itemize}
	\item	Finding prime numbers below \texttildelow10 million
\end{itemize}

\subsection{Contest Problem, All or Nothing}
After many hard days and nights, Farmer John has completed the construction of a larger barn to house his cows.
Though both parties would like to begin migration to the new barn, the cows want to perform the migration in a fair manner so that no cow gets to enjoy the new barn before another.
Each day, Farmer John plans to move an equal number of cows over.
He is not willing to do any \textit{more} work one day, nor any \textit{less} work another day, and he is \textbf{\textit{certainly}} not willing to move a single cow a day.
Therefore, in order to force Farmer John to move every cow to the new barn in a single day, the cows have decided to manipulate his head count by some amount so that he perceives that there is no viable method to have his way (that is, he cannot evenly split them up).
Find out the number of possible totals the cows can give Farmer John such that he is forced to move all of the cows to the new barn early tomorrow.

\subsubsection{Input}
You will be given the lower bound of the range of possible headcounts followed by the upper bound.
\lstinputlisting{./algorithms/sieve-of-eratosthenes/problems/sieve-of-eratosthenes.in}

\subsubsection{Output}
The number of possible head counts from the given range (inclusive).
\lstinputlisting{./algorithms/sieve-of-eratosthenes/problems/sieve-of-eratosthenes.out}

\subsubsection{Sample Solution}
\lstinputlisting{./algorithms/sieve-of-eratosthenes/problems/sieve-of-eratosthenes.cpp}

\subsubsection{Lessons Learned}
\begin{itemize}
	\item The sieve is a simple tool for finding primes $<10,000,000$.
	\item Requires a sequence of at least size N where N is equal to the upper bound (can be made more efficient by excluding even numbers).
\end{itemize}

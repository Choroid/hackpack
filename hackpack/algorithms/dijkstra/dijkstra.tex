\section{Dijkstra's Algorithm}\index{Shortest Path!Dijkstra}
Dijkstra's Algorithm solves the single-source shortest path problem of finding the shortest paths between the source 
node and all other nodes in a connected graph with non-negative edge path costs.  
The graph can be both directed and undirected.  
It is commonly used to find the shortest path between a source and destination node.  
For graphs with negative weights, see Bellman-Ford Algorithm.
It is commonly inplemented using a priority queue and runs in $O(E lg V)$.

\subsection{Applications}

\begin{itemize}
	\item  Finding the shortest paths between a source node and all other nodes in a graph
\end{itemize}

\subsection{Example Contest Problem: Farm Tour\cite{farmtour}}
When Farmer John's friends visit him on the farm, he likes to show them around. 
His farm comprises $N$ $(1 \leq N \leq 1000)$ fields numbered $1..N$, the first of which contains his house and the Nth of which contains the big barn. 
A total $M$ $(1 \leq M \leq 10000)$ paths that connect the fields in various ways. 
Each path connects two different fields and has a nonzero length smaller than $35,000$. 

To show off his farm in the best way, he walks a tour that starts at his house, potentially travels through some fields, and ends at the barn. 
Later, he returns (potentially through some fields) back to his house again. 

He wants his tour to be as short as possible, however he doesn't want to walk on any given path more than once. 
Calculate the shortest tour possible. 
Farmer John is sure that some tour exists for any given farm.

\subsubsection{Input Format}
\begin{itemize}
	\item Line $1$: Two space-separated integers: $N$ and $M$. 
	\item Lines $2..M+1$: Three space-separated integers that define a path: The starting field, the end field, and the path's length. 
\end{itemize}

\subsubsection{Sample Input}
\acmlisting[caption=Farm Tour Input, label=Farm Tour Input]{./algorithms/dijkstra/problems/farm-tour/farmtour.in}

\subsubsection{Output Format}
\begin{itemize}
	\item Line 1: A single line containing the length of the shortest tour. 
\end{itemize}

\subsubsection{Sample Output}
\acmlisting[caption=Milk Routing Output, label=Milk Routing Output]{./algorithms/dijkstra/problems/farm-tour/farmtour.out}

\subsection{Example Contest Problem: Dueling GPS's\cite{gpsduel}}
Farmer John has recently purchased a new car online, but in his haste he accidentally clicked the "Submit" button twice when selecting extra
features for the car, and as a result the car ended up equipped with two GPS navigation systems!  
Even worse, the two systems often make conflicting decisions about the route that Farmer John should take.

The map of the region in which Farmer John lives consists of N intersections $(2 \leq N \leq 10,000)$ and M directional roads $(1 \leq M \leq 50,000)$.  
Road $i$ connects intersections $A_i$ $(1 \leq A_i \leq N)$ and $B_i$ $(1 \leq B_i \leq N)$. 
Multiple roads could connect the same pair of intersections, and a bi-directional road (one permitting two-way travel) is represented 
by two separate directional roads in opposite orientations.  
Farmer John's house is located at intersection 1, and his farm is located at intersection $N$.  
It is possible to reach the farm from his house by traveling along a series of directional roads.

Both GPS units are using the same underlying map as described above; however, they have different notions for the travel time along each road. 
Road $i$ takes $P_i$ units of time to traverse according to the first GPS unit, and $Q_i$ units of time to traverse according to the second unit 
(each travel time is an integer in the range $1..100,000$).

Farmer John wants to travel from his house to the farm.  
However, each GPS unit complains loudly any time Farmer John follows a road (say, from intersection $X$ to intersection $Y$) that the GPS unit believes 
not to be part of a shortest route from $X$ to the farm (it is even possible that both GPS units can complain, 
if Farmer John takes a road that neither unit likes). 

Please help Farmer John determine the minimum possible number of total complaints he can receive if he chooses his route appropriately.  
If both GPS units complain when Farmer John follows a road, this counts as $+2$ towards the total.

\subsubsection{Input Format}
\begin{itemize}
	\item Line 1: The integers N and M. 
	\item Line i describes road i with four integers: $A_i$ $B_i$ $P_i$ $Q_i$. 
\end{itemize}

\subsubsection{Sample Input}
\acmlisting[caption=Farm Tour Input, label=Farm Tour Input]{./algorithms/dijkstra/problems/dueling-gps/gpsduel.in}

\subsubsection{Output Format}
\begin{itemize}
	\item Line 1: The minimum total number of complaints Farmer John can receive if he
        			routes himself from his house to the farm optimally.
\end{itemize}

\subsubsection{Sample Output}
\acmlisting[caption=Milk Routing Output, label=Milk Routing Output]{./algorithms/dijkstra/problems/dueling-gps/gpsduel.out}

\subsubsection{Example Solution}
\acmlisting[caption=Dueling GPSs Solution, label=Dueling GPSs Solution]{./algorithms/dijkstra/problems/dueling-gps/gpsduel.cpp}

\subsubsection{Lessons Learned}
\begin{itemize}
	\item Trickier problems may require adjusting the graph or, in this case, creating a new one
	\item Numbering the nodes starting with 0 instead of 1 allows for mapping with vector indeces
\end{itemize}

\subsection{Example Contest Problem: Milk Routing\cite{milkroute}}
Farmer John's farm has an outdated network of $M$ pipes $(1 \leq M \leq 500)$ for pumping milk from the barn to his milk storage tank.  
He wants to remove and update most of these over the next year, but he wants to leave exactly one path worth of pipes intact, 
so that he can still pump milk from the barn to the storage tank.

The pipe network is described by N junction points $(1 \leq N \leq 500)$, each of which can serve as the endpoint of a set of pipes.  
Junction point 1 is the barn, and junction point N is the storage tank.  
Each of the M bi-directional pipes runs between a pair of junction points, and has an associated latency 
(the amount of time it takes milk to reach one end of the pipe from the other) and capacity (the amount of milk per unit time
that can be pumped through the pipe in steady state).  
Multiple pipes can connect between the same pair of junction points.

For a path of pipes connecting from the barn to the tank, the latency of the path is the sum of the latencies of the pipes along 
the path, and the capacity of the path is the minimum of the capacities of the pipes along the path (since this is the "bottleneck" 
constraining the overall rate at which milk can be pumped through the path).  
If Farmer John wants to send a total of $X$ units of milk through a path of pipes with latency $L$ and capacity $C$, the time this takes is therefore $L + X/C$.

Given the structure of Farmer John's pipe network, please help him select a single path from the barn to the storage tank that will allow him to pump $X$ units
of milk in a minimum amount of total time.

\subsubsection{Input Format}
\begin{itemize}
	\item Line $1$: Three space-seperated integers: $N$ $M$ $X$ $(1 \leq X \leq 1,000,000)$.
	\item Line $2..1+M$: Each line describes a pipe using 4 integers: $I$ $J$ $L$ $C$.
			$I$ and $J$ $(1 \leq I,J \leq N)$ are the juntion points at both ends of the pipe.
			$L$ and $C$ $(1 \leq L,C \leq 1,000,000)$ give the latency and capacity of the pipe.
\end{itemize}

\subsubsection{Sample Input}
\acmlisting[caption=Milk Routing Input, label=Milk Routing Input]{./algorithms/dijkstra/problems/milk-routing/milkrouting.in}

\subsubsection{Output Format}
\begin{itemize}
	\item Line 1: The minimum amount of time it will take Farmer John to send milk along a single path, 
			rounded down to the nearest integer.
\end{itemize}

\subsubsection{Sample Output}
\acmlisting[caption=Milk Routing Output, label=Milk Routing Output]{./algorithms/dijkstra/problems/milk-routing/milkrouting.out}
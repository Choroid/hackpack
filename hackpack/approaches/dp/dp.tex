\section{Dynamic Programming}\index{Dynamic Programming}
Dynamic Programming is a powerful tool that can be applied to several different types of algorithms.\cite{dppractice}
The basic idea is to save the results of smaller problems and use the results to solve larger problems.

\subsection{Applications}
\begin{itemize}
	\item	Improving runtimes of some other algorithms
	\item	Solving the knapsack in $O(nm)$ time
	\item	Solving the integer knapsack in $O(nm)$ time
	\item	Solving the largest increasing subsequece in $O(n \log n)$ time
	\item	Solving the maximum value sub-array problem in $O(n)$
	\item	Solving the maximum value continuous sub-array problem
\end{itemize}

\subsection{Contest Problem, A Knapsack Full of Fireworks}\index{Knapsack}
The cows on Farmer John's Farm are planning a fireworks show for his birthday.

They have found all of the loose change from all over the farm and want to purchase fireworks to maximize the wow factor for their purchase.

Please help the cows determine the maximum wow factor they can get for their loose change.

\subsubsection{Input Format}
\begin{itemize}
	\item Line 1: One integer, $N$, $(1 \leq N \leq 100)$ the number of fireworks in the catalog,
	\item Line 2: One integer, $C$, $(1 \leq C \leq 10000)$ the number of cents that the cows found.
	\item Line 3..$(N+2)$ Two integers $P,W$ representing the price and wow factor for the fireworks.
\end{itemize}

\subsubsection{Sample Input}
\lstinputlisting{./approaches/dp/problems/knapsack/knapsack.in}

\subsubsection{Output Format}
\begin{itemize}
	\item Line 1: A single integer representing the maximum wow factor
\end{itemize}
\subsubsection{Sample Output}
\lstinputlisting{./approaches/dp/problems/knapsack/knapsack.out}

\subsubsection{Sample Solution}
\lstinputlisting{./approaches/dp/problems/knapsack/knapsack.cpp}

\subsubsection{Lessons Learned}
The optimal solution is of the form:
$$W(j) = \max \left\{W(j-1), \max \left\{W(j - p_i) + v_i \right\}\right\}$$
Where $W(0) = 0$

\subsection{Contest Problem, A Few Fireworks More}\index{Knapsack!Integer}
The Cows have reconsidered there original plan of buying just the fireworks with the greatest wow factor.

Instead they want to buy the most wow factor given they use each firework type at most once.

\subsubsection{Input}
\begin{itemize}
	\item Line 1: One integer, $N$, $(1 \leq N \leq 100)$ the number of fireworks in the catalog,
	\item Line 2: One integer, $C$, $(1 \leq C \leq 10000)$ the number of cents that the cows found.
	\item Line 3..$(N+2)$ Two integers $P,W$ representing the price and wow factor for the fireworks.
\end{itemize}

\subsubsection{Sample Input}
\lstinputlisting{./approaches/dp/problems/one-zero/one-zero.in}

\subsubsection{Output Format}
\begin{itemize}
	\item Line 1: A single integer representing the maximum wow factor using each firework at most once
\end{itemize}
\subsubsection{Sample Output}
\lstinputlisting{./approaches/dp/problems/one-zero/one-zero.out}

\subsubsection{Sample Solution}
\lstinputlisting{./approaches/dp/problems/one-zero/one-zero.cpp}

\subsubsection{Lesson Learned}
A similar problem to the knapsack, except each item can be used at most once.  The solution here is to expand the state space.  The optimal solution is of the form
$$M(i,j) = \max \left\{ M(i-1, j) , M(i-1, j- s_i) + v_i \right\}$$
Where $M(0,j) = 0$ and $M(i,0) = 0$

\subsection{Contest Problem, The Good, the Bad, the Cowy}\index{Largest Increasing Subsequence}
The cows have decided to put on a vote to determine if the fireworks show they put on for Farmer John's Birthday was a success.
However, being schooled in the high arts of bovine politics, a simple majority will not do as first past the post ballot systems are inherently subject to rigging.

Instead, the cows have decided to issue integer votes,
and if the largest increasing subsequence of the votes is greater than half the number of cows then the cows will determine the display a success.

Help the cows determine the results of their election.

\subsubsection{Input}
\begin{itemize}
	\item Line 1: Several Integers separated by spaces representing the votes of the cows.
\end{itemize}

\subsubsection{Sample Input}
\lstinputlisting{./approaches/dp/problems/cowy/cowy.in}

\subsubsection{Output Format}
\begin{itemize}
	\item Line 1: 1 if the cows have determined the display was a success or 0 otherwise.
\end{itemize}
\subsubsection{Sample Output}
\lstinputlisting{./approaches/dp/problems/cowy/cowy.out}

\subsubsection{Sample Solution}
\lstinputlisting{./approaches/dp/problems/cowy/cowy.cpp}

\subsubsection{Lesson Learned}
\begin{itemize}
	\item This problem can be solved in $O(n \log n)$ time
	\item Sometimes you have to check the entire array to find the solution
	\item $while( cin >> val)$ can be used to read in an uncertain number of values
\end{itemize}

\subsection{Contest Problem, Cowstock}
Farmer John loved his firework display for his birthday and has decided to reward the cows with a trip to Cowstock -- the greatest Moosiclal Concert series of all time.

Upon arriving, the cows are given a brochure containing the times and locations of all of the concerts to be given that day.
The cows want to maximize the number of concerts that they are able to attend.

Help the cows determine how may concerts they can attend.

\subsubsection{Input}
\begin{itemize}
	\item Line 1: One integer $N$, $(1<10000)$ indicating the number of concerts
	\item Line 2..$(N+1)$: $T$ the time of the song, followed by $N$ integers indicating the shortest travel time between this location and the other $N$ locations in the same order as they are presented.
\end{itemize}

\subsubsection{Sample Input}
\lstinputlisting{./approaches/dp/problems/cowstock/cowstock.in}

\subsubsection{Output Format}
\begin{itemize}
	\item Line 1: One integer indicating the number of songs that the cows get to see.
\end{itemize}
\subsubsection{Sample Output}
\lstinputlisting{./approaches/dp/problems/cowstock/cowstock.out}

\subsubsection{Sample Solution}
\lstinputlisting{./approaches/dp/problems/cowstock/cowstock.cpp}

\subsubsection{Lesson Learned}
\begin{itemize}
	\item DP is often used in conjunction with other techniques
	\item Don't be afraid to sort first and ask questions later
\end{itemize}

\section{Set}
Sets are a data structure that are useful for determining if an element has been seen before or not.
Sets are Associative, Ordered, Set, Unique Keyed, and Allocator Aware\cite{cplusplus}.
Associative means that elements are referenced by some key and not by an absolute position in the data structure.
Ordered means that internally the set follows a strict order based off the key.
Unique key means that there cannot be two keys with the same value.
Set means that the key is also the value.
Allocator-aware means that the container knows who to dynamically allocate memory for itself.
Sets are generally implemented as a binary search tree.
See "unordered set" for a version of the set that is Unordered, but is based on hash tables.

\subsection{Reference}
\lstinputlisting[language=c++]{./set/set.cpp}

\subsection{Applications}
\begin{itemize}
    \item   Determining how many and what items are in one set are also in another
    \item   Determining how many and what items are in one set but \emph{not} in another
    \item   Filtering out non-unique inputs
\end{itemize}

\subsection{Example Contest Problem: Cow Pens}
\paragraph
Farmer John's cows keep wandering off into the hills to learn cowculus from nomadic mathemeticans.
Unappreciative of refined bovine arithmetic, Farmer John decides to fence in his cows to keep them from escaping.
\paragraph
He wants to build the fence without crossing over any of the trees on his property (which the cows claim are valuable for studying graph theory), and he'd like to build the fence in the shape of a rectangle (a perfect shape, the cows say).
One side of the rectangle should be formed by the river at the southern edge of Farmer John's property, so that the cows won't complain about a lack of waves to study trigonometric functions.
\paragraph
Please compute the maximum area that Farmer John can enclose with a fence that meets the above requirements.
You can assume that the river has the positon $Y=0$ and farmer John's property lies north of the river.

\subsubsection{Input Format}
\begin{itemize}
  \item Line 1: One integer, $N$, $(N < 1000000)$, specifying the number of trees in the field.
  \item Lines 2..$1+N$: Each line contains two integers $X$ and $Y$ $(0 <= X,Y <= 1000000)$.
        Each pair corresponds to the location of one tree in Farmer John's field.
\end{itemize}

\subsubsection{Sample Input}
\lstinputlisting{./set/cowpens.in}

\subsubsection{Input Details}

\subsubsection{Output Format}
\begin{itemize}
  \item Line 1: One integer that represents the area of the largest possible rectangle that Farmer John can build.
\end{itemize}

\subsubsection{Sample Output}
\lstinputlisting{./set/cowpens.out}

\subsubsection{Output Details}

\subsubsection{Example Solution}
\lstinputlisting{./set/cowpens.cpp}

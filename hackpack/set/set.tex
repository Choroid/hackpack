\section{Set}\index{Set!Ordered Set}
Sets are a data structure that are useful for determining if an element has been seen before or not.
Sets are Associative \index{Associative}, Ordered \index{Ordered}, Set \index{Set}, Unique Keyed \index{Unique Keyed}, and Allocator Aware\index{Allocator Aware}\cite{cplusplus}.
Associative means that elements are referenced by some key and not by an absolute position in the data structure.
Ordered means that internally the set follows a strict order based off the key.
Unique key means that there cannot be two keys with the same value.
Set means that the key is also the value.
Allocator-aware means that the container knows who to dynamically allocate memory for itself.
Sets are generally implemented as a binary search tree.
See "unordered set" for a version of the set that is Unordered, but is based on hash tables.

\subsection{Reference}
\lstinputlisting[language=c++]{./set/set.cpp}

\subsection{Applications}
\begin{itemize}
    \item   Determining how many and what items are in one set are also in another
    \item   Determining how many and what items are in one set but \emph{not} in another
    \item   Filtering out non-unique inputs
\end{itemize}

\subsection{Contest Problem, The Building of Cowpens!}
Farmer John is building fences to contain his cows on his property.
However being both obsessed with right angles and not killing trees, Farmer John must construct his fence as a rectangle and not cut down any trees.
Also not wanting for his cows to be thirsty, he will build the pen along side the river on the southern side of his property.
You can assume that the river has the position $Y=0$ and farmer John's property lies north of the river.
Farmer John wants to know the area of the largest such section of his property.
\subsubsection{Input:} You will be first given the number of trees, $N<1000000$, followed by the positions, $(0,0) \leq (X,Y) \leq (1000000, 1000000)$, of the trees in the field. 
\subsubsection{Output:} The area of the largest fenced area.
\lstinputlisting{./set/cowpens.cpp}

\subsection{Contest Problem, The Cows form a Union}
The Cows have formed on a union and gone on strike!
Being faithful union members, each cow has its union id number painted on their sides.
Farmer John in efforts to track Union activity on the farm, took two pictures of the cows during their rallies: one in the morning and one at night.
He knows that the cows that stay all day with the closest id numbers are the Union bosses.
He wants to figure out which cows are the Union bosses so that he can attempt to negotiate with them.
\subsubsection{Input:} The id numbers $0 < i \leq 1000000$ of the cows that were present in photo 1 ending with 0, followed with the id numbers of the cows in photo 2 ending with 0.
\subsubsection{Output:} The id numbers of the cows with the closest id numbers in ascending order separated by a single space.

\subsection{Contest Problem, Closest Cows}
Cows often graze on grass in the feilds near Farmer John's Barn.
When looking down from the top of the silo, Farmer John ponders which pair of cows of different breeds -- one being a Holstein and one a Guernsey -- are the closest on the field.
\subsubsection{Input:} First, you will be given the number of Guernseies, $ G \leq 500000$ followed by the position $(0,0) \leq (X,Y) \leq (1000000,1000000)$ on the field of the Guernseies.
In the same way, the number of Holsteins $H \leq 500000$ will be followed by the position of the Holsteins $(0,0) \leq (X,Y) \leq (1000000,1000000)$ on the field will be given.
\subsubsection{Output:} Output the distance between the closest Holstein Guernsey pair.

\section{Flood Fill}\index{flood fill}
This algorithm is a four or eight way recursive method that checks a start node on a graph for and old value, then updates the start node value with a new value and recursively calls in four or eight directions.
When the method is called a start node location, the current node value to be changed, and the new value to update the current node value are all passed.
Inside the flood fill method the start node value is checked via if statement to not equal the old value.
Passing the if statement executes a return call ending the iteration of the method.
When the if statement is failed the start node value is updated to the new value.     
Next a series of four recursive flood fill method calls are executed in north south west and east directions from the start node location. 
The flood fill method can have eight recursive flood fill method calls to execute in all eight directions from the start node. 
The flood fill method can be implemented with both and array or a stack, as shown in the example code.
It should be noted that a target node location can also be passed to the method to have the method return located. 
This implementation creates and execution closely relating breath first search on a graph. 

\subsection{Flood Fill}
\acmlisting[caption = Flood Fill, label = Flood Fill]{./algorithms/flood-fill/examples/flood-fill.cpp}

\subsection{Flood Fill with Stack}
\acmlisting[caption = Flood Fill with Stack, label = Flood Fill with Stack]{./algorithms/flood-fill/examples/flood-fill-stack.cpp}

\subsection{Flood Fill with Target Node}
\acmlisting[caption = Flood Fill with Target Node, label = Flood Fill with Stack]{./algorithms/flood-fill/examples/flood-fill-target.cpp}

\subsection{Input Example}
\acmlisting[caption = Input Example, label = Input Example]{./algorithms/flood-fill/examples/flood-fill.in}

\subsection{Output Example}
\acmlisting[caption = Output Example, label = Output Example]{./algorithms/flood-fill/examples/flood-fill.out}

\section{Max Flows}\index{graph!Max Flows}
There are several types of problems that involve finding maximum flow in a graph.
Many of these problems can easily tackled by making small transformations to the graph.


If there is more than one source or sink,  one can add a "super" source and or sink that connects to all of the sources and sinks.
Then find the flow from the super source to the super sink.


If the graph is undirected, simply insert every edge twice in both directions.


If the flow is limited through the nodes instead of the edges, simply spit every node into a "in" node and an "out" node.



\begin{enumerate}
	\item Greedily
\end{enumerate}

In a directed acyclic graph, this can be done in linear time.

\begin{enumerate}
	\item Topologically sort the graph using a breadth first search
	\item Use dynamic programming to build up the maximum flow.
\end{enumerate}

\subsection{Applications}
\subsection{Example Contest Problem: Cow Connection}
\subsubsection{Input Format}
\subsubsection{Sample Input}
\acmlisting[label=cowex sample input, caption=cowex sample input]{./algorithms/max-flow/problems/cowex/cowex.in}
\subsubsection{Output Format}
\subsubsection{Sample Output}
\acmlisting[label=cowex sample output, caption=cowex sample output]{./algorithms/max-flow/problems/cowex/cowex.out}
\subsubsection{Example Solution}
\acmlisting[label=cowex sample output, caption=cowex sample output]{./algorithms/max-flow/problems/cowex/cowex.cpp}
\subsubsection{Lessons Learned}
\begin{itemize}
	\item The inner most while loop is a linear solution to the widest most path problem.
\end{itemize}

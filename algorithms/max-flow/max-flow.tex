\section{Max Flows}\index{graph!Max Flows}
#ifdef hackpackpp
There are several types of problems that involve finding maximum flow in a graph.
Many of these problems can easily tackled by making small transformations to the graph.


If there is more than one source or sink, one can add a "super" source and or sink that connects to all of the sources and sinks.
Then find the flow from the super source to the super sink.


If the graph is undirected, simply insert every edge twice in both directions.


If the flow is limited through the nodes instead of the edges, simply split every node into an "in" node and an "out" node.

In a general directed graph, this can be done in $O(FE)$ where F is the maximum flow and E is the number of edges.
\begin{itemize}
	\item While there is a path from source to sink
	\begin{itemize}
		\item Greedily find the widest path
		\item Reduce the cost of the nodes on the widest path by its width
		\item Create reverse paths by the width of the widest path
	\end{itemize}
\end{itemize}

In a directed acyclic graph, this can be done in linear time.

\begin{itemize}
	\item Topologically sort the graph using a breadth first search
	\item Use dynamic programming to build up the maximum flow.
\end{itemize}
#endif

\subsection{Applications}
\begin{itemize}
	\item Finding the maximum flow in a graph
	\item Finding a minimum set of edges required to disconnect source and sink
	\item Finding a maximal matching
\end{itemize}
#ifdef hackpackpp

\subsection{Example Contest Problem: Cow-Ex}
The cows are up to another of their insidious schemes.
The cows have setup a milk distribution system on Farmer John's farm.

The pipes that the cows use to transport the milk are of various sizes.
Each pipe has been rated with a positive integer referring to the amount of milk that the pipe can carry in 1 minute.

The cows want to know how much milk they can transport to the barn at once.
\subsubsection{Input Format}
\begin{itemize}
	\item Line 1: A positive integer N that represents the number of pipes that the cows have in place. $0 < N \leq 10000$
	\item Line 2..$(N+1)$: Three non-negative integers, $S$, $D$, $E$ representing a pipe going from location $S$ to location $D$ with a capasity of $E$. $0 \leq S,D < N$
	\item Line $(N+2)$: A non-negative integer $A$, that represents location where the milk is produced
	\item Line $(N+3)$: A non-negative ingeger $B$, that represents location of the barn.
\end{itemize}

\subsubsection{Sample Input}
\acmlisting[label=Cowex Sample Input, caption=Cowex Sample Input]{./algorithms/max-flow/problems/cowex/cowex.in}

\subsubsection{Output Format}
\begin{itemize}
	\item Line 1: A Positive integer representing the maximum capacity of the network.
		This should be 0 if it is not possible to transport the milk.
\end{itemize}
\subsubsection{Sample Output}
\acmlisting[label=Cowex Sample Output, caption=Cowex Sample Output]{./algorithms/max-flow/problems/cowex/cowex.out}
\subsubsection{Example Solution}
#endif

#ifdef hackpack
\subsection{Greedy Max Flow Algorithm}
#endif

\acmlisting[label=Cowex Sample Solution, caption=Cowex Sample Solution]{./algorithms/max-flow/problems/cowex/cowex.cpp}
#ifdef hackpackpp
\subsubsection{Lessons Learned}
\begin{itemize}
	\item The inner most while loop is a solution to the widest most path problem that runs in $O(V)$ time.
\end{itemize}
#endif

